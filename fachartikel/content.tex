\hypertarget{abstract}{%
\section{Abstract}\label{abstract}}

In der vorliegenden Arbeit wurde untersucht, wie ein Arduino Uno über
eine WebSocket-Verbindung gesteuert werden kann, während dieser im
Sekundentakt Statusmeldungen versendet.

Der verwendete Funduino Uno R3 \autocite{lernset} verfügt über keine
WLAN-Schnittstelle. Daher erfolgt die WebSocket-Kommunikation extern auf
einem Gateway. Als Gateway wird ein Windows-PC verwendet. Arduino und
Gateway sind per USB verbunden und kommunizieren über eine virtuelle
serielle Schnittstelle.

Das Ergebnis zeigt, dass eine Steuerung eines Arduinos per WebSocket
problemlos möglich ist. Die eingehenden Nachrichten müssen auf dem
Arduino geparst werden. Dieses Parding ist einfacher, wenn das
verwendete Austauschformt schlank gehalten wird. Deshalb wurde anstelle
von JSON das URL Format verwendet.

\hypertarget{einleitung}{%
\section{Einleitung}\label{einleitung}}

In diesem Kapitel wird die Motivation erläutert und genaue Fragesellung
definiert. Dann folgt eine kleine Übersichtsarbeit mit dazugehöriger
Literaturrecherche.

\hypertarget{fragestellung}{%
\subsection{Fragestellung}\label{fragestellung}}

Welche Möglichkeiten gibt es, einen Arduino Uno via Websocket zu
steuern?

Während einer explorativen Online-Suche wurden einzelne Lösungen
gefunden. Eine systematische Zusammenstellung der Möglichkeiten fehlt
jedoch.

\hypertarget{motivation}{%
\subsection{Motivation}\label{motivation}}

Die Motivation für die vorliegende Arbeit ist die Beantwortung der
nachfolgenden Fragestellung. Weiter soll der Artikel interessierten
Lesern als Einstiegslektüre diesen.

\hypertarget{literatur-review}{%
\subsection{Literatur-Review}\label{literatur-review}}

Zum Thema existiert diverse Fachliteratur unter anderem von Erik
Bartmann\autocite{bartmannArduino}\autocite{bartmannESP8266}\autocite{bartmannESP32}.

\hypertarget{arduino-mit-integriertem-wlan}{%
\subsubsection{Arduino mit integriertem
WLAN}\label{arduino-mit-integriertem-wlan}}

Der Arduino Uno hat keine eingebaute WLAN Schnittstelle. Es gibt jedoch
andere Arduino Modelle mit integriertem WLAN, wie z.B. der Arduino
MKR1000.

\hypertarget{wlan-erweiterung}{%
\subsubsection{WLAN Erweiterung}\label{wlan-erweiterung}}

Mehrere Autoren
berichten\autocite{temperatureDashboard}\autocite{websocketcommunication},
wie der Arduino mit dem dem WLAN Modul ESP8266 erweitert werden kann.

\hypertarget{serial-gateway}{%
\subsubsection{Serial Gateway}\label{serial-gateway}}

Eine weitere Möglichkeit, ist behelfsweise einen PC als Serial Gateway
einzusetzen. Mangels kurzfristig verfügbarer Hardware wollen wir diese
Option verfolgen.

\hypertarget{experimenteller-teil}{%
\section{Experimenteller Teil}\label{experimenteller-teil}}

\hypertarget{informationsquellen}{%
\subsection{Informationsquellen}\label{informationsquellen}}

Als Informationsquellen sind die Datenblätter zur jweiligen Hardware
sowie die Manuals zu den einsesetzten Softwarekomponenten zu nennen.

\hypertarget{prinzipskizze}{%
\subsection{Prinzipskizze}\label{prinzipskizze}}

\hypertarget{hardware}{%
\subsection{Hardware}\label{hardware}}

Verwendet wurde das Lernset Nr. 8 von Funduino\autocite{lernset}. Darin
enthalten ist ein Funduino Uno. Weiter benötigen wir den
Temparatursensor TMP36 und den Fotowiderstand.

\hypertarget{anschluss-der-sensoren}{%
\subsubsection{Anschluss der Sensoren}\label{anschluss-der-sensoren}}

\hypertarget{software}{%
\subsection{Software}\label{software}}

\hypertarget{entwicklungsumgebung}{%
\subsubsection{Entwicklungsumgebung}\label{entwicklungsumgebung}}

Zur Entwicklung wurde folgende Software eingesetzt.

\begin{itemize}
\item
  Visual Studio Code\autocite{vscode} mit der Erweiterung C/C++
  IntelliSense\autocite{intellisense}
\item
  Arduino CLI\autocite{arduinoCli}
\item
  Git for Windows\autocite{gitForWindows} und
  TortoiseGit\autocite{tortoiseGit}
\end{itemize}

Nicht verwendet wurde die Arduino IDE. Windows verwendet den
Standardtreiber \emph{usbser.sys} für den virtuellen COM Port.

\hypertarget{node-libraries}{%
\subsubsection{Node Libraries}\label{node-libraries}}

Weiter wurde folgende NPM Packages eingesetzt:

\begin{itemize}
\item
  WebSockets \autocite{websockets}
\item
  Express \autocite{express}
\item
  Chart.js \autocite{chartjs}
\item
  SerialPort \autocite{serialPort}
\end{itemize}

\hypertarget{arduino-libraries}{%
\subsubsection{Arduino Libraries}\label{arduino-libraries}}

Weiter wurde folgende Arduino Libraries eingesetzt:

\begin{itemize}
\item
  Arduino Library (Arduino.h)
  \autocite{sprachreferenz}\autocite{codeReferenz}\autocite{arduinoCheatSheet}
\item
  AVR Libc \autocite{avrlibc}
\end{itemize}

\hypertarget{arduino-sketch}{%
\subsubsection{Arduino Sketch}\label{arduino-sketch}}

Zunächst müssen wir klären, in welcher Programmiersprache die Arduino
Sketches geschrieben werden. Nachdem man sich die Build-Umgebung genauer
unter die Lupe genommen hat, wird klar, dass keine eigene
Arduino-Sprache existert\autocite{arduinoLanguage}. Im Hintergrund wird
aus dem Sketch eine C++ Datei erstellt und mit \emph{avr-g++}
kompiliert.

Die Problematik der Heap-Fragmentierung wird von mehreren Autoren
aufgeworfen und diskutiert \autocite{heapFragmentation}
\autocite{heapFragmentation2}. Matt ist der Meinung, dass man deshalb
auf die String Klasse in der Arduino Library gänzlich verzichten
soll\autocite{arduinoStrings}. In der Konsequenz müsste man die
Stringfunktion aus der Standard C Library\autocite{avrlibc} verwerden
und in C programmieren. Ich sehe dies nicht ganz so eng und setze die
Arduino String Klasse trotzdem, jedoch mit Zurückhaltung ein. Ich
befolge Matt's Rat, die Variablen by Reference zu
übergeben\autocite{arduinoStrings}.

Der Quellcode befindet sich im Anhang.

\hypertarget{serial-gateway-1}{%
\subsubsection{Serial Gateway}\label{serial-gateway-1}}

Der Quellcode befindet sich im Anhang.

\hypertarget{websocket-server}{%
\subsubsection{WebSocket Server}\label{websocket-server}}

Der Quellcode befindet sich im Anhang.

\hypertarget{web-gui}{%
\subsubsection{Web GUI}\label{web-gui}}

Der Quellcode befindet sich im Anhang.

\hypertarget{resultate}{%
\section{Resultate}\label{resultate}}

Es hat sich gezeigt, dass ein Seriell-zu-Websocket-Gatway unter Node.js
einfach zu implementieren ist. Über diesen Umweg kann der Arduino Uno
ans Internet angebunden werden.

\hypertarget{diskussion}{%
\section{Diskussion}\label{diskussion}}

\hypertarget{zusammenfassung}{%
\section{Zusammenfassung}\label{zusammenfassung}}

Statt des Arduino Uno könnte ein Arduino MKR1000 verwendet werden.
Dieser könnte kann auch an die Arduino Clound angebunden werden. Ein
weitere Option ist die Beschaffung einer WLAN Erweiterung wie das Modul
ESP8266.

\hypertarget{danksagung}{%
\section{Danksagung}\label{danksagung}}

Ich danke den Lernenden der Klasse BINF2017A für die Zusammenarbeit.

\hypertarget{interessenskonflikte}{%
\section{Interessenskonflikte}\label{interessenskonflikte}}

Das Projekt wurde im Rahmen des Beruffachschulunterrichts durchgeführt
und erhielt keine externde Finanzierung. Demnach bestehen keien
Interessenkonflikte.
