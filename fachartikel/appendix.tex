\appendix{}
\section{Quellcode}
\subsection{Arduino Sketch} \label{Arduino Sketch}
\lstinputlisting[language=c++]{../arduino/genericReadWrite/genericReadWrite.ino}
\subsubsection{Funktionen zur Statusausgabe}
\lstinputlisting[language=c++]{../arduino/genericReadWrite/status.cpp}
\subsubsection{Funktionen zur Steuerung}
\lstinputlisting[language=c++]{../arduino/genericReadWrite/control.cpp}

\newpage
\subsection{Serial Gateway} \label{Serial Gateway}
\lstinputlisting[language=js]{../gateway/gateway.js}

\newpage
\subsection{WebSocket Server} \label{WebSocket Server}
\lstinputlisting[language=js]{../websocketserver/websocketserver.js}

\newpage
\subsection{Web Client} \label{Web Client}
\lstinputlisting[language=js]{../client/index.html}

\newpage
\section{Erstellung dieses Dokuments}
Dieses Dokument wurde mit LaTeX erstellt. Der Quellcode befindet sich in der Datei
\verb|_doc/paper.tex|. Die Quellensammlung befindet sich in der Datei \verb|_doc/quellen.tex| klk

\subsection{LaTeX Umgebung}
Dazu wurde TeX Live \cite{texlive} installiert. Dieses
Paket stellt zahlreiche Tools zur Verfügung, darunter  \verb|pdflatex|

\subsection{LaTeX Editor}
Als Editor wurde Visual Studio Code \cite{vscode} mit der Erweiterung LaTeX Workshop \cite{latexWorkshop} verwendet.

Das automatische Speichern kann in Visual Studio Code unter \verb|File -> Auto save| aktiviert werden. Mit \verb|Ctrl+Alt+V| wird die Vorschau des generierten PDFs aktiviert. Mit \verb|Alt+Z| kann der automatische Zeilenumbruch eingeschalten werden.

Vorteile gegenüber Microsoft Word sind die automatische Vervollständigung beim Schreiben, sowie die einfache Einbindung von Quellcode Listings wie in Anhang A.

Weitere Informationen zum Erstellen des Inhaltsverzeichnis und des Literaturverzeichnis finden sich unter \cite{inhaltsverzeichnis} und \cite{literaturverzeichnis}.

\subsection{LaTeX Pakete}
Wie in \verb|_doc/setup.tex| zu sehen, werden folgende Pakete verwendet.
\begin{itemize}
    \item babel -- Multilingual support for Plain TeX or LaTeX
    \item xcolor -- Driver-independent color extensions for LaTeX and pdfLaTeX
    \item listings -- Typeset source code listings using LaTeX
    \item biblatex -- Sophisticated Bibliographies in LaTeX
    \item csquotes -- Context sensitive quotation facilities
\end{itemize}
Diese Pakete sind im Lieferumfang von TeX Live dabei und müssen nicht speziell installiert werden.