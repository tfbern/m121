\section{Experimenteller Teil}

\subsection{Informationsquellen}
Als Informationsquellen sind die Datenblätter zur jweiligen Hardware sowie die Manuals zu den einsesetzten Softwarekomponenten zu nennen.

\subsection{Prinzipskizze}

\subsection{Hardware}
Verwendet wurde das Lernset Nr. 8 von Funduino\cite{lernset}. Darin enthalten ist ein Funduino Uno. Weiter benötigen wir den Temparatursensor TMP36 und den Fotowiderstand.

\subsubsection{Anschluss der Sensoren}

\subsection{Software}

\subsubsection{Entwicklungsumgebung}
Zur Entwicklung wurde folgende Software eingesetzt.
\begin{itemize}
    \item Visual Studio Code\cite{vscode} mit der Erweiterung C/C++ IntelliSense\cite{intellisense}
    \item Arduino CLI\cite{arduinoCli}
    \item Git for Windows\cite{gitForWindows} und TortoiseGit\cite{tortoiseGit} 
  \end{itemize}
Nicht verwendet wurde die Arduino IDE. Windows verwendet den Standardtreiber \textit {usbser.sys} für den virtuellen COM Port. 

\subsubsection{Node Libraries}
Weiter wurde folgende NPM Packages eingesetzt:
\begin{itemize}
  \item WebSockets \cite{websockets}
  \item Express \cite{express}
  \item Chart.js \cite{chartjs}
  \item SerialPort \cite{serialPort}
\end{itemize}

\subsubsection{Arduino Libraries}
Weiter wurde folgende Arduino Libraries eingesetzt:
\begin{itemize}
  \item Arduino Library (Arduino.h) \cite{sprachreferenz}\cite{codeReferenz}\cite{arduinoCheatSheet}
  \item AVR Libc \cite{avrlibc}
\end{itemize}

\subsubsection{Arduino Sketch}
Zunächst müssen wir klären, in welcher Programmiersprache die Arduino Sketches geschrieben werden. Nachdem man sich die Build-Umgebung genauer unter die Lupe genommen hat, wird klar, dass keine eigene Arduino-Sprache existert\cite{arduinoLanguage}. Im Hintergrund wird aus dem Sketch eine C++ Datei erstellt und mit \textit{avr-g++} kompiliert.

Die Problematik der Heap-Fragmentierung wird von mehreren Autoren aufgeworfen und diskutiert \cite{heapFragmentation}\cite{heapFragmentation2}. Matt ist der Meinung, dass man deshalb auf die String Klasse in der Arduino Library gänzlich verzichten soll\cite{arduinoStrings}. In der Konsequenz müsste man die Stringfunktion aus der Standard C Library\cite{avrlibc} verwerden und in C programmieren. Ich sehe dies nicht ganz so eng und setze die Arduino String Klasse trotzdem, jedoch mit Zurückhaltung ein. Ich befolge Matt's Rat, die Variablen by Reference zu übergeben\cite{arduinoStrings}.

Der Quellcode befindet sich im Kapitel \ref{Arduino Sketch}.

\subsubsection{Serial Gateway}
Der Quellcode befindet sich im Kapitel \ref{Serial Gateway}.

\subsubsection{WebSocket Server}
Der Quellcode befindet sich im Kapitel \ref{WebSocket Server}.

\subsubsection{Web GUI}
Der Quellcode befindet sich im Kapitel \ref{Web Client}.